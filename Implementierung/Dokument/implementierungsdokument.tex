\documentclass[a4paper]{scrreprt}
\usepackage[german]{babel}
\usepackage[utf8]{inputenc}
\usepackage{graphicx}
\usepackage{pdflscape}

\begin{document}
\title{Implementierungsdokument}
\author{Hanselmann, Hecht, Klein, Schnell, Stapelbroek, Wohnig}
\maketitle 
\tableofcontents	
\listoffigures


\chapter{Einleitung}
Dieses Dokument beschreibt die Implementierungsphase einer Praxis der Softwareentwicklungsgruppe am Karlsruher Institut für Technologie. Der Titel der Gruppenaufgabe lautet: \textit{Entwicklung eines Werkzeugs zur Analyse formaler Eigenschaften von Wahlverfahren}. \\
Diese Dokument stellt die in dieser Phase entstandenen Unterschiede zu den vorherigen Phasen (Pflichtenheft und Entwurf) dar und erklärt, warum diese notwendig wurden. \\
Weiterhin wird die zeitliche sowie die personelle Aufteilung der Implementierung vorgestellt. \\



\chapter{Unterschiede zu den im Pflichtenheft gestellten Kriterien}


Eventuell auch Trennung nach GUI-Fenstern, obwohl das vermutlich weniger sinnvoll ist.
\section{Musskriterien}
Falls Ein Kriterium nicht implementiert wurde, ausführlich begründen warum nicht.

\section{Sollkritierien}
\section{Kannkriterien}
Falls Ein Kriterium nicht implementiert wurde, begründen warum nicht. 

\chapter{Änderungen am Entwurf}
Bspw. \\
Änderung abstrakte Fabrik in highlevel ist jetzt keine Abstrakte Fabrik mehr.\\ Dieses Entwurfsmuster konnte nicht verwendet werden, da die zu erstellenden Objekte teilweise voneinander abhängig sind. Weiterhin gibt es Objekte, die mehrere Rollen einnehmen, d.h. sie implementieren unterschiedliche Interfaces. \\
Die Lösung des Problems bietet ein so von uns genannter CentralObjectProvider der, bei seiner eigenen Konstruktion alle anderen Highlevelobjekte baut und diese dann über get zu Verfügung stellt.
Somit hat der BEASTCommunicator immer noch eine abstrakte Sicht auf alle Interfaces.

\chapter{Zeitablauf Implementierungsphase}

\section{Geplanter Ablauf}
\section{Eigentlicher Ablauf}
Lines of Code\\
Graphiken dazu, wer was gemacht hat.
Vielleicht eine Graphik über die Commits.


\end{document}