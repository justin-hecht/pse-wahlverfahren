\documentclass[a4paper]{report}
\usepackage{scrextend}
\usepackage[german]{babel}
\usepackage[utf8]{inputenc}
\usepackage[T1]{fontenc}
\usepackage{ae}
\usepackage[bookmarks,bookmarksnumbered]{hyperref}
%graphics
\usepackage{graphicx}
\graphicspath{ {images/GUI/c-editor/}
 {images/GUI/Eigenschaften-Liste/}
  {images/GUI/Eigenschaften/}
  {images/GUI/Parameter-editor/}} 

\usepackage[printonlyused]{acronym}
\usepackage{hyperref}
\usepackage{float}

%glossary stuff, must be defined after hyperref
\usepackage[toc]{glossaries}
\makeglossaries
\newglossaryentry{CBMC}
{
name={C bounded model checker},
description={Ein Programm, welches C-Programme mit bounded model checking auf Fehler und vom user definierte assertions untersucht}
}
\newglossaryentry{Makro}
{
name={Makro},
description={Ein Codeabschnitt, der durch einen Präprozessor ersetzt wird.}
}



\begin{document}

\title{Pflichtenheft}
\author{Beste Gruppe}
\maketitle
 

\tableofcontents	

\listoffigures

\chapter*{Abkürzungsverzeichnis}
\begin{acronym} %put all your abbreviations here 
\acro{CBMC}{C Bounded Model Checker}
\acro{BMC}{Bounded Model Checking}
\acro{GUI}{Graphical User Interface}
\end{acronym}
 
\chapter{Produktübersicht}
Wahlverfahren bilden den Grundstein unserer Demokratie. Dabei werden viele Anforderungen an sie gestellt, welche unsere intuitiven Ideen über Gerechtigkeit formalisieren: Proportionalität, Anonymität, etc. Moderne Wahlverfahren sind oft so komplex, dass sie viele überraschende und teils unerwünschte Eigenschaften haben. Nachweisen deren Abwesenheit ist absolut nicht trivial. So wurde beispielsweise 2008 das Bundestagswahlrecht vom BVerfG für verfassungswidrig erklärt, da es unter anderem die Gleichheit der Wirkung verschiedener Stimmen verletzte. Auf der anderen Seite ist es auch sehr schwer, Wahlverfahren auf die Präsenz erwünschter Eigenschaften zu untersuchen.

\ac{BMC} wird normalerweise dazu verwendet zu überprüfen, ob ein gegebenes Programm gegebene Eigenschaften erfüllt. Da dieses Problem im Allgemeinen unentscheidbar ist, werden nur endliche Codepfade überprüft. Dadurch wird der Zustandsraum endlich und das Problem entscheidbar. Um dies zu bewerkstelligen, werden potentiell unendliche Codepfade - also Schleifen - bis zu einer vom Benutzer bestimmten Grenze aufgerollt. Danach wird eine SAT-Formel erstellt, die erfüllbar ist, genau dann wenn das Programm einen Zustand einnehmen kann, welcher die gegebene Eigenschaft nicht erfüllt. Dies ist vollautomatisch und gibt bei Nichterfüllung das Gegenbeispiel zurück.  

In unserem Fall kann \ac{BMC} konkret dazu verwendet werden, ein C-Programm darauf zu untersuchen ob es im Falle gegebener Vorbedingungen gegebene Nachbedingungen erfüllt. Dies wird dazu verwendet, obige Problemstellung innerhalb einer bestimmten Genauigkeit zu lösen: so kann ein in C beschriebenes Wahlverfahren wie z.B. die einfache Mehrheitswahl darauf geprüft werden, ob es bestimmte Eigenschaften erfüllt. Allerdings ist es kompliziert, dies direkt zu tun. 

Unser Programm ist im Wesentlichen eine sehr umfangreiche Schnittstelle um mit \ac{CBMC} zu kommunizieren. Es bietet dem Benutzer über eine \ac{GUI} die Möglichkeiten, formale Eigenschaften für Wahlverfahren sowie diese Wahlverfahren selbst anzugeben und zu editieren. Weiterhin liefert es Möglichkeiten, die Interaktion mit \ac{CBMC} zu gestalten: Für wie viele Wähler, Plätze etc die Eigenschaft überprüft werden soll. Nach erfolgreicher Überprüfung durch \ac{CBMC} bekommt der Benutzer schließlich eine Antwort des Programms, in der er bei Nichterfüllung der Eigenschaft ein Gegenbeispiel angezeigt bekommt. Wird kein Gegenbeispiel gefunden, so wird eine Erfolgsmeldung ausgegeben. All dies wird graphisch über die \ac{GUI} aufbereitet.

Die \ac{GUI} ist nach Funktionalität in vier Teilen angeordnet:
\begin{enumerate}
\item "`C-Editor"': Code-Editor für Wahlverfahren in der Programmiersprache C
\item "Eigenschaften-Liste": Listenansicht aller Eigenschaften, die für dieses Wahlverfahren untersucht werden sollen
\item "`Eigenschaften-Editor"': Editor für Spezifikation formaler Eigenschaften als boolsche Ausdrücke in eigens dafür vorgesehener Grammatik
\item "`Params"': Eingabe von Parametern einer zu analysierenden Wahl
\end{enumerate}

\section{Die Sprache zur Angabe der formalen Eigenschaften} \label{Sprache-für-formale-Eigenschaften}
In diesem Abschnitt wird ein grober Überblick über die Sprache, welche der Eigenschaften-Editor verwendet, gegeben. Es handelt sich um ein Subset der C-Sprache mit einigen Ergänzungen. Diese werden im Folgenden erläutert. 

Formale Eigenschaften werden in Vor- und Nachbedingungen unterteilt. Diese wiederum werden vom User als eine Liste boolscher Ausdrücke angegeben. Die Sprache erlaubt folgende Konstrukte zur Darstellung boolscher Ausdrücke:

\begin{itemize}
\item Das logische Und (\verb!&&!), Oder (\verb!&&!), Implikation (\verb!==>!), Äquivalenz (\verb!<==>!), Gleichheit (\verb!==!), Ungleichheit (\texttt{!=}), kleiner als (\texttt{<}), kleiner gleich (\texttt{<=}), größer als (\texttt{>}) und größer gleich (\texttt{>=})\\
Beispiel: \verb!x > y <==> x + 1 > y + 1! \\
Bedeutung: \verb!x! is größer als \verb!y! genau dann, wenn auch \verb!x + 1! größer als \verb!y + 1! ist  	
\item Symbolische Variablen vom Typ Wähler, Kandidat oder Sitz. Für deren Benennung gelten dieselben Regeln wie für die Benennung von Variablen in C \\
Beispiel: \verb!Wähler v, Kandidat c!
\item Quantoren für Wähler, Kandidaten und Sitze in der Form von \gls{Makro}s. Ein bisher ungenutzter Variablenname wird als Argument erwartet. Dieser kann in dem darauf folgenden Ausdruck als symbolische Variable entsprechenden Typs verwendet werden. \\
Beispiel: \verb|FOR_ALL_VOTERS(v) : EXISTS_ONE_CANDIDATE(c) : v mag c| \\
Bedeutung: Für jeden Wähler (\verb!v!) gibt es zumindest einen Kandidaten, welchen er 'mag' (Dies ist nur ein Beispiel, der Editor wird die Eigenschaft 'mögen' nicht zu Verfügung stellen)
\item Ausgabe der Anzahl Stimmen für einen Kandidaten in der Form eines Makros
\item Viele Eigenschaften benötigen zu ihrer Überprüfung das Vergleichen mehrerer Wahlen. Dies wird ermöglicht durch Variablen VOTESx() und ELECTx. Dabei steht 'x' für die Nummer des Wahldurchangs. VOTESx() erwartet als Argument eine symblische Variable vom Typ Wähler. Zurück gibt es die Stimme, welche v im Wahldurchgang x abgegeben hat. ELECTx erwartet kein Argument und gibt das Wahlergebnis im x-ten Durchgang zurück. Der Rückgabetyp beider Makros hängt von der Kategorie des Wahlverfahrens ab. Gibt das Wahlverfahren zum Beispiel nur einen "Gewinner" aus, so ist ELECTx vom Typ Kandidat. \\
Beispiel: \verb|FOR_ALL_VOTERS(v) : VOTES1(v) == VOTES2(v)| \\
Bedeutung: Alle Wähler wählen in beiden Wahlen (\verb!VOTES1! und \verb!VOTES2! ) gleich. \\
Beispiel: \verb|ELECT1 == ELECT2|
Bedeutung: Das Ergebnis der ersten Wahl stimmt mit dem der zweiten überein
\item Folgende Konstanten: Anzahl Wähler (\verb!V!), Anzahl Kandidaten (\verb!C!) und Anzahl Sitze (\verb!S!)
\end{itemize}
Beendet wird ein boolscher Ausdruck mit einem Semikolon. 


\chapter{Zielbestimmung}
Ziel des Programmes ist es eine  Lösung zur Untersuchung formaler Eigenschaften von Wahlverfahren zu präsentieren, welche auch von Nicht-Informatikern mit minimalem Aufwand erlernt und eingesetzt werden kann. 
Es soll Folgendes bereitstellen:
\begin{itemize}
\item Eine Möglichkeit zur Beschreibung diverser Wahlverfahren in C-Code  
\item Eine Möglichkeit zur Beschreibung der formalen Eigenschaften, welche das Wahlverfahren erfüllen soll, in der beschriebenen Sprache zur Angabe der formalen Eigenschaften
\item Eine Möglichkeit zum Angeben der Parameter (Anzahl Wähler, Anzahl Kandidaten, Anzahl Sitze) 
\item Eine Ausgabe des Ergebnisses der Überprüfung: eine Erfolgsmeldung bei Erfolg und Präsentation eines Gegenbeispiels bei Nichterfolg
\end{itemize}

Die Überprüfung der gegebenen Eigenschaften wird durch den \ac{CBMC} geschehen. Aufgabe des Programmes wird es sein, die gegebenen Eingaben für den \ac{CBMC} aufzubereiten, sowie dessen Ausgabe zu interpretieren und zu präsentieren. 

All diese Aufgaben ließen sich theoretisch auch schon jetzt, ohne Verwendung unseres Programms erledigen. Allerdings wäre der damit verbundene Lern- und Einarbeitungsufwand sehr hoch, vor allem bei der Angabe der formalen Eigenschaften. Weiterhin ist damit viel, sich jedes Mal wiederholender Aufwand, verbunden, welcher sich automatisieren lässt. Daher ist ein Schwerpunkt unseres Programmes einfache Benutzbarkeit, besonders für Nicht-Informatiker. Dies soll erreicht werden über eine \ac{GUI}, welche oft benötigte Funktionalität bereitstellt. Einfache syntaktische Fehler im Code sollen während des Editierens erkannt werden. Dadurch soll das Untersuchen von Wahlverfahren leichter und schneller werden, was den Mehrwert unseres Programmes ausmacht.

\section{Musskriterien}
\begin{itemize}
\item Das Programm kann auf 32-Bit Versionen von Windows und Linux-Betriebssystemen betrieben werden
\item Alle Abhängigkeiten werden mit dem Programm ausgeliefert
\item Das Programm bietet einen Code-Editor für das zu prüfende Wahlverfahren
	\begin{itemize}
	\item Der Code kann abgespeichert und geladen werden
	\item Der Code-Editor zeigt Fehler im eingegebenen Code an
	\item Aktionen können widerrufen und wiederhergestellt werden
	\end{itemize}
\item Es können formale Eigenschaften zur Prüfung des Wahlverfahrens eingegeben werden
	\begin{itemize}
	\item Eine formale Eigenschaft kann abgespeichert und geladen werden	
	\item Fehler in der Eingabe werden angezeigt
	\end{itemize}
\item Die Parameter der Wahl (Anzahl von Wählern, Kandidaten und Sitzen) können festgelegt werden
\item Das Ergebnis der Überprüfung wird vom Programm angezeigt. Im Fall der Verletzung einer formalen Eigenschaft wird ein Gegenbeispiel vom Programm angezeigt
\end{itemize}


\section{Sollkriterien}
\begin{itemize}
\item Die Parameter der Wahl können in Intervallen angegeben werden.
\item Der Code-Editor bietet folgende Funktionalitäten:
	\begin{itemize}
	\item Syntax-Highlighting
	\item Automatisches Einrücken
	\item Tastatur-Shortcuts
	\item Codevorlagen
	\end{itemize}
\item Code completion bei der Eingabe der formalen Eigenschaften ist möglich
\item Die Analyse des Wahlverfahrens kann abgebrochen werden
\end{itemize}


\section{Wunschkriterien}
\begin{itemize}
\item Das Programm kann auf einem Mac betrieben werden
\item Der Code-Editor bietet folgende Funktionalitäten:
	\begin{itemize}
	\item Code completion
	\item Warnung vor nicht unterstützten Elementen der Programmiersprache wie z.B. Threads
	\end{itemize}
\item Es kann festgelegt werden, wie lange die Analyse des Wahlverfahrens maximal dauern soll
\item Eine Wahl kann eingegeben werden. Das Ergebnis wird angezeigt
\end{itemize}

\section{Abgrenzungskriterien}
\begin{itemize}
\item Das Programm kann keine Angabe darüber machen, wie Lange die Überprüfung einer Eigenschaft dauern wird
\item Es wird nicht bestätigt, ob ein gegebenes Wahlverfahren eine formale Eigenschaft erfüllt
\end{itemize}



\chapter{Produkteinsatz}
Das Programm überprüft Wahlverfahren auf ihre formalen Eigenschaften. Es richtet sich an Kunden, die ein Interesse an der Erforschung oder Entwicklung solcher Verfahren haben. Grundsätzlich sollte das Programm aber auch für Nicht-Informatikern verständlich sein. Für die Bedienung des Programms ist jedoch Kenntnis der Programmiersprache C und der Aussagen- und Prädikatenlogik nötig.

\section{Anwendungsbereiche}
\begin{itemize}
\item Universitärer Bereich
\item Forschung
\end{itemize}

\section{Zielgruppen}
\begin{itemize}
\item Wahlforscher
\item Softwareentwickler
\item Hobbyisten
\end{itemize}

\section{Betriebsbedingungen}
Das Produkt kommt in einer Büroumgebung zum Einsatz. Es wird auf einem aktuellen Computer mit aktuellen Werten für Arbeitsspeicher und Rechnergeschwindigkeit betrieben.


\chapter{Produktumgebung}

\section{Software}
\begin{itemize}
\item Das Betriebssystem ist entweder Microsoft Windows (7 oder moderner) oder eine der Linux-Distributionen Arch oder Ubuntu lauffähig.
\end{itemize}

\section{Hardware}
\begin{itemize}
\item PC mit Tastatur und Maus
\end{itemize}

\section{Produkt-Schnittstellen}
Über das Produkt wird \ac{CBMC} angesteuert.

\chapter{Funktionale Anforderungen}
\section{Allgemein}
/F10/ Bereitstellen von Editoren zur Beschreibung des Wahlverfahrens sowie zur Beschreibung zu erfüllender formaler Eigenschaften \\
/F20/ Kommunikation und Überprüfung dieser Eigenschaften via \ac{CBMC} \\
/F30/ Bereitstellen von Kommunikationsschnittstellen mit \ac{CBMC} sowohl für Eingabe von Parametern als auch Ausgabe der Ergebnisse, welche auch für Nicht-Informatiker verständlich ist \\
/F40/ Möglichkeit des Speicherns von Code, formaler Anforderungen und Eingabeparametern als ein Projekt \\
/F50/ Möglichkeit des Öffnens der Projekte aus /F40/

\section{C-Code Editor für Wahlverfahren}
\subsection{Muss-Kriterien}
/FM10/ Darstellung aller für das Programmieren in C benötigten Charaktere \\
/FM20/ Veränderung des dargestellten Textes durch Eingabe anderer Charaktere über die Tastatur wie in Notepad \\
/FM30/ Speichern von erstelltem Code als Datei auf der Festplatte an vom User angegebenen Ort \\
/FM40/ Laden und Darstellen von Dateien korrekten Formats \\
/FM50/ Anzeigen syntaktischer Fehler in dem Fehler Fenster (siehe \ref{Editor-mit-Fehler-nach-statischer-analyse})

\subsection{Soll-Kriterien}
/FS10/ Überprüfen des Formats beim Ladevorgang. Falls falsches Format: Ausgabe einer Fehlermeldung \\
/FS20/ Syntax-Highlighting: Darstellung diverser Schlüsselwörter in anderen Farben als den Rest des Codes. Dies beinhaltet, ist jedoch nicht beschränkt auf: 
\begin{itemize}
\item Typendeklaration (int, float, ...)
\item Kontrollflow-Konstrukte (if, else, while...)
\item Variablennamen
\item Kommentare
\end{itemize}
/FS30/ Anzeigen der Zeilennummern am linken Zeilenrand \\
/FS40/ Anzeigen von Syntaktischen Fehlern im Code, welche durch einen Lexer oder Parser erkannt werden können: 
\begin{itemize}
\item Verwendung von Schlüsselwörtern als Variablennamen 
\item Vergessene Semikolons am Ende von Anweisungen
\item Nicht geschlossene Klammern und Anführungszeichen
\item Andere Konstrukte, welche der Grammatik der C-Sprache widersprechen
\end{itemize}

/FS50/ Reaktion auf typische Tastenkürzel\\
\begin{table}[H]
\caption{Hotkeys und verbundene Operationen}
\begin{tabular}{lcr} 
Kürzel & Operation \\
\hline 
Strg + c & Kopieren \\
Strg + x & Auschneiden \\
Strg + v & Einfügen \\
Strg + z & Zuletzt ausgeführte Aktion Rückgängig machen \\
Strg + r & Zuletzt rückgängig gemachte Aktion erneut ausführen \\
Strg + s & Speichern \\
Strg + o & Öffnen \\
Strg + Leer & Anzeigen der Code-Completion Vorschläge\\
\end{tabular}
\label{table:Hotkeys_and_operations}
\end{table}

/FS60/ Bereitstellen von Wahl-Templates
\begin{itemize}
\item Jeder Wähler wählt genau einen Kandidaten
\item Jeder Wähler ordnet Kandidaten nach Präferenz in absteigender Reihenfolge 
\item Jeder Wähler ordnet Kandidaten eine Nummer zwischen MAX (maximale Zustimmung) und MIN (maximale Abneigung) zu. MAX und MIN sind dabei vom User konfigurierbar. 
\end{itemize}

\subsection{Kann-Kriterien}
/FK10/ Automatisches Einrücken des Codes in Schleifen und if-Statements \\
/FK20/ Code-Completion
\begin{itemize}
\item Automatisches Schließen von Klammern und Anführungszeichen
\item Primitiv: Vorschlagen bereits im Code vorgekommener Wörter
\item Intelligent: Durch Analysieren eines ASTs nur Vorschlagen der Wörter welche im Kontext Sinn ergeben.
\end{itemize}

/FK30/ Durch den User konfigurierbares Verhalten:
\begin{itemize}
\item Festlegen der Farben, welche beim Syntax-Highlighting verwendet werden
\item Festlegen des verwendeten Fonts
\item An- und Ausschalten der angezeigten Zeilennummern
\item Festlegen wie vielen Leerzeichen ein Tab entspricht
\end{itemize}
 
\section{Editor für formale Eigenschaften}
\subsection{Muss-Kriterien}
/FM10/ Darstellung aller für das Programmieren in C benötigten Charaktere \\
/FM20/ Veränderung des dargestellten Textes durch Eingabe anderer Charaktere über die  Tastatur wie in Notepad \\
/FM21/ Beschreibung formaler Eigenschaften als Vor- und Nachbedingung als boolsche Ausdrücke (siehe \ref{Sprache-für-formale-Eigenschaften})\\
/FM30/ Bereitstellung von Makros zur Beschreibung der Eigenschaften (siehe \ref{table:Macros_for_formal_Attributes}) \\

\begin{table}[H]
\caption{Makros zur Beschreibung formaler Eigenschaften}
\begin{tabular}{|p{5cm}|p{10cm}|}
\hline 
Makro & Effekt \\
\hline 
\verb!FOR_ALL_VOTERS(i)! & In der darauf folgenden Eigenschaft kann i als symbolische Variable verwendet werden. Gesamtausdruck ist wahr falls sie für alle Wähler gilt \\
\hline 
\verb!FOR_ALL_CANDIDATES(i)! & In der darauf folgenden Eigenschaft kann i als symbolische Variable verwendet werden. Gesamtausdruck ist wahr falls sie für alle Kandidaten gilt \\
\hline 
\verb!FOR_ALL_SEATS(i)! & In der darauf folgenden Eigenschaft kann i als symbolische Variable verwendet werden. Gesamtausdruck ist wahr falls sie für alle Sitze gilt \\
\hline 
\verb!EXISTS_ONE_VOTER(i)! & In der darauf folgenden Eigenschaft kann i als symbolische Variable verwendet werden. Gesamtausdruck ist wahr falls sie für mindesten einen Wähler gilt \\
\hline 
\verb!EXISTS_ONE_CANDIDATE(i)! & In der darauf folgenden Eigenschaft kann i als symbolische Variable verwendet werden. Gesamtausdruck ist wahr falls sie für mindesten einen Kandidaten gilt \\
\hline 
\verb!EXISTS_ONE_SEAT(i)! & In der darauf folgenden Eigenschaft kann i als symbolische Variable verwendet werden.Gesamtausdruck ist wahr falls sie für mindesten einen Sitz gilt \\
\hline 
\verb!VOTE_SUM_FOR_CANDIDATE(c)! & Gibt die Anzahl Stimmen für Kandidaten c zurück\\
\hline 
\end{tabular}
\label{table:Macros_for_formal_Attributes}
\end{table}

/FM40/ Bereitstellen symbolischer Variablen für Wähler, Kandidaten und Sitze \\
/FM50/ Bereitstellen von Operatoren für Implikation und Äquivalenz \\
/FM51/ Bereitstellen von Operatoren für logisches UND, ODER, GLEICH und NICHT GLEICH \\
/FM52/ Bereitstellen von Vergleichen: Kleiner, Kleiner gleich, größer, größer gleich \\
/FM60/ Beliebig tiefe, lediglich von Hardware begrenzte, Schachtelung dieser Konstrukte \\

\subsection{Soll-Kriterien}
/FS10/ Syntax-Highlighting \\
/FS20/ Anzeigen von Syntaktischen Fehlern im Code \\

\subsection{Kann-Kriterien}
/FK10/ Code-Completion
\begin{itemize}
\item Auto-Vervollständigung der Makros
\item Analyse des Codes und Anzeigen relevanter, bereits definierter Eigenschaften und symbolischer Variablen
\end{itemize}

\section{Editor für Eingabeparameter}
/F10/ Möglichkeit zur Angabe der zu analysierenden Anzahl von Wählern, Kandidaten und Sitzen \\
/F20/ Möglichkeit zum Eingeben einer Zeitspanne nach welcher die Berechnung abgebrochen wird \\

\section{Ausgabe der Analyseergebnisse}
/F10/ Ausgabe einer Erfolgsmeldung bei Erfolg \\
/F20/ Darstellung eines Gegenbeispiels bei Misserfolg\\ 

\chapter{Produktdaten}
\section{Code-Editor Wahlverfahren}
/D10/ Das Wahlverfahren ist als Methode "`unsigned int voting(params)"' einer C-Headerdatei definiert und wird mit der Endung .h gespeichert.

\section{Editor von formalen Eigenschaften}
/D20/ Die formale Eigenschaft, derer das Wahlverfahren genügen soll, ist als C-Datei definiert, die einmal die Methode voting(params) aus einer Headerdatei aufruft, und wird mit der Endung .c gespeichert.

\section{Parameter}
/D30/ Angegebene Parameter für Wahlen werden in einer Textdatei gespeichert.

\section{Projektdaten}
/D40/ Ein Projekt wird als Liste von Dateien in einer Textdatei gespeichert.


\chapter{Nichtfunktionale Anforderungen}
/F10/ Nicht mehr als 0,5 Sekunden Verzögerung bei Erfragen der Code-Completion


\chapter{Globale Testfälle}
\section{Testfälle für den C-Code Editor für Wahlverfahren}
/T110/ Erstellen eines neuen Dokuments \\
/T120/ Speichern des C-Codes 
\begin{addmargin}[25pt]{0pt} 
/T121/ Speichern \\
/T122/ Speichern unter
\end{addmargin}
/T130/ Laden von C-Code aus einer Datei \\
/T140/ Auswahl des zu verwendenden Wahl-Templates \\
/T150/ Verwendung der Funktionen Ausschneiden, Einfügen, Kopieren, Rückgängig und Wiederholen \\
/T160/ Änderung der Eigenschaften des Editors \\
/T170/ Editieren des C-Codes \\
/T180/ Statische Analyse des Codes \\

\section{Testfälle für die Eigenschaften-Liste}
/T210/ Eine neue Liste erstellen \\
/T220/ Eine Liste speichern 
\begin{addmargin}[25pt]{0pt} 
/T221/ Speichern \\
/T222/ Speichern unter
\end{addmargin}
/T230/ Eine Liste öffnen \\
/T240/ Ablesen des Ergebnisses der Überprüfung der formalen Eigenschaft. \\
/T250/ Einstellen ob eine formale Eigenschaft, die in der Liste geladen ist, in der \\ nächsten Überprüfung verwendet werden soll. \\
/T260/ Die Funktionen Rückgängig und Wiederholen verwenden \\
/T270/ Eine neue formale Eigenschaft in die Liste aufnehmen \\
\section{Testfälle für den Editor für formale Eigenschaften}

/T310/ Eingabe und Editieren einer formalen Eigenschaft 
\begin{addmargin}[25pt]{0pt}
/T311/ Eingabe von Vor- und Nachbedingungen \\
/T312/ Verwendung der Bereitgestellten Makros \\
/T313/ Verwendung der symbolischer Variablen 
\end{addmargin}	
/T320/ Eine Formale Eigenschaft speichern
\begin{addmargin}[25pt]{0pt}
/T321/ Speichern \\
/T322/ Speichern unter 
\end{addmargin}	
/T330/ Laden einer formalen Eigenschaft aus einer Datei \\
/T340/ Verwendung der Funktionen Ausschneiden, Einfügen, Kopieren, Rückgängig und \\ Wiederholen
\section{Testfälle für den Editor für Eingabeparameter}

/T410/ Eingabe der Anzahl von Wählern, Kandidaten und Sitzen
\begin{addmargin}[25pt]{0pt}
/T411/ Eventuell erfolgt die Eingabe in Intervallen
\end{addmargin}	
/T420/ Eingabe einer maximalen Zeitspanne \\
/T430/ Starten und stoppen der Überprüfung des Wahlverfahren auf die gewählte formale Eigenschaft(en) \\
/T440/ Eine komplettes Projekt speichern 
\begin{addmargin}[25pt]{0pt}
/T441/ Speichern \\
/T442/ Speichern unter 
\end{addmargin}
/T450/ Eine komplettes Projekt laden \\
/T460/ Eine neues Projekt anlegen \\


\chapter{Systemmodelle}
\section{Szenarien}

\textbf{Szenario 1}\\
Ein Wahlforscher wird von der Regierung eines Landes beauftragt ein neues Wahlverfahren zu entwerfen. Dieses soll bestimmte, vorgegebene formale Eigenschaften erfüllen.\\
Er installiert das Tool und entwickelt mit diesem eine erste Version des Wahlverfahrens. Zusätzlich gibt er die vorgegebenen Eigenschaften als formale Eigenschaften in das Tool ein.\\
Nun gibt er Parameter zum Prüfen an (Wähler, Stimmen, Sitze) und lässt das Wahlverfahren auf die formalen Eigenschaften prüfen.\\
Das Tool gibt aus, dass nicht alle Eigenschaften vom Wahlverfahren erfüllt werden.\\
Mithilfe der Informationen aus der Ergebnisausgabe analysiert und modifiziert der Entwickler das Wahlverfahren so, dass es alle Eigenschaften erfüllt.\\
Er prüft es nochmals auf alle Eigenschaften und nun werden alle erfüllt.\\
Zuletzt speichert er Wahlverfahren und formale Eigenschaften ab und sendet ersteres an eine Prüfstelle für Wahlverfahren.\\
\\
\textbf{Szenario 2}\\
Eine Prüfstelle bekommt ein Wahlverfahren übergeben welches sie auf bestimmte formale Eigenschaften testen soll.\\
Das Tool wird installiert und das Wahlverfahren im C-Editor geladen und es werden formale Eigenschaften erstellt.\\
Um das Wahlverfahren auf die Eigenschaften zu prüfen werden Parameter zu \\Kandidaten-, Stimmen- und Wähleranzahl, sowie einem Timeout als obere Zeitgrenze für das Prüfen angegeben.\\
Wenn das Tool fertig mit dem Überprüfen ist, werden erfüllte und nichterfüllte Eigenschaften angezeigt. Diese Informationen benutzt die Prüfstelle dann in ihrem Bericht über das Wahlverfahren.
\pagebreak
\section{Anwendungsfälle}
Der Nutzer (Entwickler, Wahlforscher, Prüfstelle..) kann Wahlverfahren und formale Eigenschaften in den jeweiligen Editoren des Tools entwickeln bzw. editieren.
Beide Editoren verfügen über Optionen zum Laden und Speichern, sowie dem Anzeigen von Fehlern und Syntax-Highlighting.\\
Es kann geprüft werden ob das Wahlverfahren die formalen Eigenschaften erfüllt. Voraussetzung dafür sind ein fehlerfreies Wahlverfahren und fehlerfreie formale Eigenschaften. Vor dem Prüfen kann man Testparameter angeben (Anzahl von Kandidaten, Wählern, Sitzen und einem Timeout, dass dem Prüfen eine zeitliche Obergrenze gibt).\\
Nach dem Prüfen wird für jede Eigenschaft angezeigt ob sie vom Wahlverfahren erfüllt wird oder nicht. Werden formale Eigenschaften als nicht erfüllt erkannt, werden Gegenbeispiele angezeigt die die Nichterfüllbarkeit beweisen. Anhand dieser Informationen kann der Nutzer dann das Wahlverfahren und die Eigenschaften analysieren und falls nötig modifizieren.


\hspace{-2cm}\includegraphics[scale=0.09]{UseCaseDiagram.png}

	


\chapter{GUI}
Die hier vorgestellte \ac{GUI} erfüllt alle Muss-, Soll- und Kann-Kriterien. Das endgültige Produkt kann daher davon abweichen. Im Folgenden wird jedes mal darauf hingewiesen, falls es sich bei einem Feature um ein Kann-Kriterium handelt.
Die \ac{GUI} besteht aus 4 verschiedenen Fenstern: 
\begin{itemize}
\item Ein Editor für C-Code, in welchem die Wahlverfahren editiert werden können
\item Eine Liste, in welcher alle für dieses Wahlverfahren zu überprüfenden Eigenschaften angezeigt werden
\item Ein Editor in welchem Eigenschaften editiert werden können
\item Das Hauptfenster, dessen Schließen ein Beenden des kompletten Tools nach sich zieht. Darin können Parameter für Überprüfungen eingestellt und Überprüfungen gestartet bzw. beendet werden
\end{itemize}
Jedes dieser Elemente verfügt auch über weitere Eigenschaften, die im Folgenden beschrieben werden.

\section{C-Editor}
Der C-Editor verfügt über dieselbe Funktionalität, welche andere Texteditoren wie zum Beispiel Notepad aufweisen. Ziel ist es, das Eingeben von Funktionen, welche ein Wahlverfahren implementiert, zu ermöglichen. Dazu bietet er die Möglichkeit, C-Code zu schreiben und zu bearbeiten. Ein angemessener Funktionskörper, welcher die auswählbare Art der Wahl, repräsentiert, wird dabei automatisch generiert (siehe \ref{Editor-mit-text}). Es wird nicht möglich sein, außerhalb dieser Funktion zu editieren. Dies untersagt, aufgrund der Gegebenheiten von C, zum Beispiel selbst definierte Funktionen. Während des Eingebens des Codes wird dieser automatisch analysiert, um Schlüsselwörter sowie syntaktische Fehler zu markieren. 
Der C-Editor teilt sich in vier Untereinheiten auf: Der Menüstreifen, die Tool-Leiste, das Textfeld und das Fehlerfeld. Der Menü-Streifen ist unterteilt in Datei, Bearbeiten, Editor (Kann) und Code. Bilder aller geöffneten Untermenüs befinden sich im Anhang. Sie beinhalten folgende Funktionalität:

\begin{table}[H]
\begin{tabular}{|p{3cm}|p{12cm}|}
Menüpunkt & Bedeutung \\
\hline
Neu & öffnet ein neues Dokument, wobei die Art der Wahl vom User angegeben wird \\
Speichern & speichert das Dokument unter bereits gegebenem Namen \\
Speichern unter & Speichert das Dokument unter neuem Namen an neuem Ort, beide durch User angegeben \\
Öffnen & Öffnet neues Dokument des richtigen Formats
\end{tabular}
\label{Datei-Menüpunkte}
\caption{Unterpunkte des Datei-Menüs}
\end{table}

\begin{table}[H]
\begin{tabular}{|p{3cm}|p{12cm}|}
Menüpunkt & Bedeutung \\
\hline
Rückgängig & Falls möglich: Macht die letzte ausgeführte Aktion Rückgängig \\
Wiederholen & Wiederholt die zuletzt Rückgängig gemachte Aktion \\
Kopieren & Fügt markierten Text in die Zwischenablage ein \\
Ausschneiden & Fügt markierten Text in die Zwischenablage ein und entfernt ihn aus dem Textfeld \\
Einfügen & Fügt Text aus der Zwischenablage an der Stelle des Cursors ein \\
Wahlart ändern & Ändert den Funktionskörper zu dem der vom User ausgewählten Art
\end{tabular}
\label{Bearbeiten-Menüpunkte}
\caption{Unterpunkte des Bearbeiten-Menüs}
\end{table}

\begin{table}[H]
\begin{tabular}{|p{3cm}|p{12cm}|}
Menüpunkt & Bedeutung \\
\hline
Einstellungen & Öffnet den Einstellungen-Dialog. Dies ist Teil der Kann-Kriterien. Falls implementiert, wird es Möglichkeiten zur Einstellung des Fonts und Syntax-Highlighting geben.
\end{tabular}
\label{Editor-Menüpunkte}
\caption{Unterpunkte des Editor-Menüs}
\end{table}

\begin{table}[H]
\begin{tabular}{|p{3cm}|p{12cm}|}
Menüpunkt & Bedeutung \\
\hline
Statische Analyse & Startet eine statische Analyse des Codes, welche ihn auf von Lexer oder Parser erkennbare Fehler untersucht. Gefundene Fehler werden in dem Fehlerfeld angezeigt. Zusätzliches Kann-Kriterium: Die Zeile, in welcher der Fehler ist, wird zusätzlich durch einen roten Punkt markiert (siehe \ref{Editor-mit-Fehler-nach-statischer-analyse}
\end{tabular}
\label{Editor-Menüpunkte}
\caption{Unterpunkte des Code-Menüs}
\end{table}

Über den Tool-Streifen lassen sich einige dieser Aktionen ohne Öffnen eines Menüs ausführen. Von Links nach Rechts: Neu, Rückgängig, Wiederholen, Speichern, Speichern unter, Öffnen, Kopieren, Ausschneiden, Einfügen.

In \ref{Editor-mit-text} sieht man den Editor nach Eingabe diverser Elemente der C-Sprache. Anzeige der Zeilennummern ist Soll-Kriterium, Möglichkeit diese Funktion zu deaktivieren Kann-Kriterium. Die grauen Balken zeigen an, dass man nur den Bereich in dem vorgegebenen Funktionskörper editieren kann.

\begin{figure}[H]
\includegraphics[scale=0.5]{Editor-ohne-text.png}
\caption{Der C Editor ohne Code}
\end{figure}

\begin{figure}[H]
\includegraphics[scale=0.5]{Editor-mit-text.png}
\caption{Der C Editor mit Code und Anzeige der Wahlart}
\label{Editor-mit-text}
\end{figure}

\begin{figure}[H]
\includegraphics[scale=0.5]{Editor-mit-Fehler-ohne-hover.png}
\caption{Fehleranzeige bei syntaktischem Fehler ohne Maus-Hover (Kann-Kriterium)}
\label{Editor-mit-Fehler-ohne-hover}
\end{figure}

\begin{figure}[H]
\includegraphics[scale=0.5]{Editor-mit-Fehler-mit-hover.png}
\caption{Fehleranzeige bei syntaktischem Fehler mit Maus-Hover (Kann-Kriterium)}
\label{Editor-mit-Fehler-mit-hover}
\end{figure}

In \ref{Editor-mit-Fehler-ohne-hover} sieht man, wie der gefundene syntaktische Fehler während des Editieren des Codes angezeigt wird. Dies ist ein Kann-Kriterium. Sobald man mit der Maus über die markierte Stelle geht, wird in einem neuen Fenster nahe der Maus eine Beschreibung des Fehlers angezeigt. Dies ist ebenfalls Kann-Kriterium (siehe \ref{Editor-mit-Fehler-mit-hover}).

\begin{figure}[H]
\includegraphics[scale=0.5]{Editor-mit-Fehler-nach-statischer-analyse.png}
\caption{Fehleranzeige nach statischer Analyse}
\label{Editor-mit-Fehler-nach-statischer-analyse}
\end{figure}

\ref{Editor-mit-Fehler-nach-statischer-analyse} Zeigt die Anzeige der Fehler nach ausführend einer statischen Code Analyse. Markierung der Zeile ist Kann-Kriterium.

\section{Eigenschaften-Liste}
Die GUI trennt das Editieren der zu überprüfenden Eigenschaften (in eigens zu diesem Zweck erstellter Syntax, siehe \ref{Sprache-für-formale-Eigenschaften}) und das Zuordnen dieser Eigenschaften zu Wahlverfahren. Dadurch können diese Eigenschaften einzeln abgespeichert und flexibel wiederverwertet und kombiniert werden. Das Zuordnen zu Wahlverfahren geschieht in der Eigenschaften-Liste. Darin werden die einzelnen Eigenschaften namentlich aufgelistet (siehe ). Im Folgenden werden die einzelnen \ac{GUI}-Bestandteile und der Funktionalität erläutert.

\begin{table}[H]
\begin{tabular}{|p{3cm}|p{12cm}|}
Menüpunkt & Bedeutung \\
\hline
Neu & Startet eine neue Liste \\
Speichern & Speichert die Liste \\
Speichern unter & Speichert die Liste unter neuem Namen an neuem Ort \\
Öffnen & öffnet Liste (Ort von User angegeben)
\end{tabular}
\label{Eigenschaftenliste-Datei-Menüpunkte}
\caption{Unterpunkte des Datei-Menüs}
\end{table}

\begin{table}[H]
\begin{tabular}{|p{3cm}|p{12cm}|}
Menüpunkt & Bedeutung \\
\hline
Rückgängig & Falls möglich: Macht die letzte ausgeführte Aktion Rückgängig \\
Wiederholen & Wiederholt die zuletzt Rückgängig gemachte Aktion 
\end{tabular}
\label{Eigenschaftenliste-Bearbeiten-Menüpunkte}
\caption{Unterpunkte des Bearbeiten-Menüs}
\end{table}

Der Tool-Streifen hat exakt den selben Zweck wie der des Code-Editors, ohne die Funktionen Kopieren, Ausschneiden und Einfügen.

\begin{figure}[H]
\begin{minipage}{.5\textwidth}
  \centering
  \includegraphics[scale=0.5]{nach-testen.png}
  \caption{•}{figure}{Liste nach Überprüfung}
  \label{fig:sub1}
\end{minipage}
\begin{minipage}{.5\textwidth}
  \centering
  \includegraphics[scale=0.5]{gegenbeispiel.png}
  \caption{figure}{Anzeige des Gegenbeispiels}
  \label{fig:sub2}
\end{minipage}
\end{figure}

Es folgt eine Beschreibung der Icons, welche zu sehen sind.

\begin{table}[H]
\begin{tabular}{|p{3cm}|p{12cm}|}
Icon & Bedeutung \\
\hline
Pfeil nach rechts & Falls Gegenbeispiel gefunden: Durch Klicken öffnet sich unter dem Listenelement ein Textfeld in welchen das Gegenbeispiel dargestellt wird \\
Checkbox & Nur falls aktiviert, wird das Wahlverfahren auf die Eigenschaft getestet \\
Maulschlüssel & Öffnet den Eigenschaften-Editor für die Eigenschaft\\
Rotes Kreuz & Entfernt die Eigenschaft aus der Liste \\
Gründes Plus & Fügt neue, leere Eigenschaft oder bereits gespeicherte Eigenschaft der Liste hinzu
\end{tabular}
\label{Eigenschaftenliste-Bearbeiten-Menüpunkte}
\caption{Icons der Eigenschaften-Liste}
\end{table}

\section{Eigenschaften Editor}

\begin{figure}[H]
\includegraphics[scale=0.5]{raw-ohne-code.png}
\caption{Eigenschaften-Editor ohne Code mit symbolischen Variablen}
\label{Eigenschaften-Editor-ohne-code}
\end{figure}

Der Eigenschaften-Editor hat den Zweck, das Bearbeiten der zu überprüfenden Eigenschaften zu ermöglichen. Eigenschaften werden aufgeteilt in Vor- und Nachbedingungen. Möglich ist die Verwendung sowohl von Quantoren in der Form von Makros als auch symbolischer Variablen. 

Die Untermenüs Datei, Bearbeiten und Code sowie der Tool-Streifen sind analog zu denen des C-Editors. Hinzu kommen jedoch Konstanten und Makros. Bereitgestellte Konstanten sind: 

\begin{itemize}
\item Die Anzahl der Wähler (V)
\item Die Anzahl der Kandidaten (C)
\item Die Anzahl vorhandener Sitze (S)
\end{itemize}

Bereitgestellte Makros sind:

\begin{itemize}
\item \verb!FOR_ALL_VOTERS()!
\item \verb!FOR_ALL_CANDIDATES()!
\item \verb!FOR_ALL_SEATS()!
\item \verb!EXISTS_ONE_VOTER()!
\item \verb!EXISTS_ONE_CANDIDATES()!
\item \verb!EXISTS_ONE_SEAT()!
\item \verb!SUM_VOTES_FOR_CANDIDATE()!
\end{itemize}

Jedes dieser Makros bis auf das Letzte nimmt eine bisher ungenutzte Variable als Argument. Diese kann im darauf Folgenden boolschen Ausdruck verwendet werden (siehe \ref{Eigenschaften-Editor-Anonymität}). Das Letzte Makro nimmt eine bereits definierte symbolische Variable vom Typ Kandidaten und gibt berechnet die Anzahl Stimmen für diesen Kandidaten.

\begin{figure}[H]
\includegraphics[scale=0.5]{Editor-vor-und-nachbedingungen-syntax-highlighting.png}
\caption{Eigenschaften-Editor mit Beispielhafter Eigenschaft Anonymität und beispielhaft dargestelltem Syntax-Highlighting (Kann-Kriterium)}
\label{Eigenschaften-Editor-Anonymität}
\end{figure}

Wie man zusätzlich sehen kann, sind auch Implikationen (\verb!==>!) möglich. Auch das logische und (\verb!&&!), oder (\verb!||!) und die Äquivalenz (\verb!<==>!) werden zur Verfügung stehen. Die Anzeige erkannter Fehler wird analog zum C-Editor geschehen.

\section{Parameter Editor}

Der Parameter Editor stellt das Hauptfenster der Anwendung dar. Es erlaubt das Einstellen und Starten der Tests. Hier lassen sich auch komplette Projekte Speichern - Code, Eigenschaften und Parameter gebündelt.

\begin{figure}[H]
\includegraphics[scale=1]{Parameter-editor.png}
\caption{Der Parameter Editor. PERS steht für Professional Election Rigging System}
\label{Parameter-editor}
\end{figure}

Das Datei Menü ist analog zu dem des C Editors. Das Projekt Menü wird im Folgenden erläutert. 

\begin{table}[H]
\begin{tabular}{|p{5cm}|p{10cm}|}
Menüpunkt & Bedeutung \\
\hline
Teste Eigenschaften & Startet Tests mit den angegebenen Parametern \\
Stop Test & Unterbricht momentan laufenden Test
\end{tabular}
\label{Parameter-Projekt-Menü}
\caption{Parameter Editor Projekt Menüpunkte}
\end{table}

Der Pfeil und das Stopp-Zeichen des Tool-Streifens haben ebenfalls die Wirkungen Teste Eigenschaften und Stop Test. 

Als Mögliche Zeitangaben stehen Sekunden, Minuten, Stunden und Tage zur Verfügung.

\chapter{Phasenverantwortliche}
\section{Pflichtenheft} Justin Hecht
\section{Entwurf} Holger Klein 
\section{Implementierung} Niels Hanselmann, Nikolai Schnell
\section{Qualitätssicherung} Lukas Stapelbroek
\section{Abschlusspräsentation} Jonas Wohnig


\printglossaries
 

 
\end{document}
